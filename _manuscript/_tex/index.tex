% Options for packages loaded elsewhere
\PassOptionsToPackage{unicode}{hyperref}
\PassOptionsToPackage{hyphens}{url}
\PassOptionsToPackage{dvipsnames,svgnames,x11names}{xcolor}
%
\documentclass[
  authoryear,
  preprint]{elsarticle}

\usepackage{amsmath,amssymb}
\usepackage{iftex}
\ifPDFTeX
  \usepackage[T1]{fontenc}
  \usepackage[utf8]{inputenc}
  \usepackage{textcomp} % provide euro and other symbols
\else % if luatex or xetex
  \usepackage{unicode-math}
  \defaultfontfeatures{Scale=MatchLowercase}
  \defaultfontfeatures[\rmfamily]{Ligatures=TeX,Scale=1}
\fi
\usepackage{lmodern}
\ifPDFTeX\else  
    % xetex/luatex font selection
\fi
% Use upquote if available, for straight quotes in verbatim environments
\IfFileExists{upquote.sty}{\usepackage{upquote}}{}
\IfFileExists{microtype.sty}{% use microtype if available
  \usepackage[]{microtype}
  \UseMicrotypeSet[protrusion]{basicmath} % disable protrusion for tt fonts
}{}
\makeatletter
\@ifundefined{KOMAClassName}{% if non-KOMA class
  \IfFileExists{parskip.sty}{%
    \usepackage{parskip}
  }{% else
    \setlength{\parindent}{0pt}
    \setlength{\parskip}{6pt plus 2pt minus 1pt}}
}{% if KOMA class
  \KOMAoptions{parskip=half}}
\makeatother
\usepackage{xcolor}
\setlength{\emergencystretch}{3em} % prevent overfull lines
\setcounter{secnumdepth}{5}
% Make \paragraph and \subparagraph free-standing
\makeatletter
\ifx\paragraph\undefined\else
  \let\oldparagraph\paragraph
  \renewcommand{\paragraph}{
    \@ifstar
      \xxxParagraphStar
      \xxxParagraphNoStar
  }
  \newcommand{\xxxParagraphStar}[1]{\oldparagraph*{#1}\mbox{}}
  \newcommand{\xxxParagraphNoStar}[1]{\oldparagraph{#1}\mbox{}}
\fi
\ifx\subparagraph\undefined\else
  \let\oldsubparagraph\subparagraph
  \renewcommand{\subparagraph}{
    \@ifstar
      \xxxSubParagraphStar
      \xxxSubParagraphNoStar
  }
  \newcommand{\xxxSubParagraphStar}[1]{\oldsubparagraph*{#1}\mbox{}}
  \newcommand{\xxxSubParagraphNoStar}[1]{\oldsubparagraph{#1}\mbox{}}
\fi
\makeatother

\usepackage{color}
\usepackage{fancyvrb}
\newcommand{\VerbBar}{|}
\newcommand{\VERB}{\Verb[commandchars=\\\{\}]}
\DefineVerbatimEnvironment{Highlighting}{Verbatim}{commandchars=\\\{\}}
% Add ',fontsize=\small' for more characters per line
\usepackage{framed}
\definecolor{shadecolor}{RGB}{241,243,245}
\newenvironment{Shaded}{\begin{snugshade}}{\end{snugshade}}
\newcommand{\AlertTok}[1]{\textcolor[rgb]{0.68,0.00,0.00}{#1}}
\newcommand{\AnnotationTok}[1]{\textcolor[rgb]{0.37,0.37,0.37}{#1}}
\newcommand{\AttributeTok}[1]{\textcolor[rgb]{0.40,0.45,0.13}{#1}}
\newcommand{\BaseNTok}[1]{\textcolor[rgb]{0.68,0.00,0.00}{#1}}
\newcommand{\BuiltInTok}[1]{\textcolor[rgb]{0.00,0.23,0.31}{#1}}
\newcommand{\CharTok}[1]{\textcolor[rgb]{0.13,0.47,0.30}{#1}}
\newcommand{\CommentTok}[1]{\textcolor[rgb]{0.37,0.37,0.37}{#1}}
\newcommand{\CommentVarTok}[1]{\textcolor[rgb]{0.37,0.37,0.37}{\textit{#1}}}
\newcommand{\ConstantTok}[1]{\textcolor[rgb]{0.56,0.35,0.01}{#1}}
\newcommand{\ControlFlowTok}[1]{\textcolor[rgb]{0.00,0.23,0.31}{\textbf{#1}}}
\newcommand{\DataTypeTok}[1]{\textcolor[rgb]{0.68,0.00,0.00}{#1}}
\newcommand{\DecValTok}[1]{\textcolor[rgb]{0.68,0.00,0.00}{#1}}
\newcommand{\DocumentationTok}[1]{\textcolor[rgb]{0.37,0.37,0.37}{\textit{#1}}}
\newcommand{\ErrorTok}[1]{\textcolor[rgb]{0.68,0.00,0.00}{#1}}
\newcommand{\ExtensionTok}[1]{\textcolor[rgb]{0.00,0.23,0.31}{#1}}
\newcommand{\FloatTok}[1]{\textcolor[rgb]{0.68,0.00,0.00}{#1}}
\newcommand{\FunctionTok}[1]{\textcolor[rgb]{0.28,0.35,0.67}{#1}}
\newcommand{\ImportTok}[1]{\textcolor[rgb]{0.00,0.46,0.62}{#1}}
\newcommand{\InformationTok}[1]{\textcolor[rgb]{0.37,0.37,0.37}{#1}}
\newcommand{\KeywordTok}[1]{\textcolor[rgb]{0.00,0.23,0.31}{\textbf{#1}}}
\newcommand{\NormalTok}[1]{\textcolor[rgb]{0.00,0.23,0.31}{#1}}
\newcommand{\OperatorTok}[1]{\textcolor[rgb]{0.37,0.37,0.37}{#1}}
\newcommand{\OtherTok}[1]{\textcolor[rgb]{0.00,0.23,0.31}{#1}}
\newcommand{\PreprocessorTok}[1]{\textcolor[rgb]{0.68,0.00,0.00}{#1}}
\newcommand{\RegionMarkerTok}[1]{\textcolor[rgb]{0.00,0.23,0.31}{#1}}
\newcommand{\SpecialCharTok}[1]{\textcolor[rgb]{0.37,0.37,0.37}{#1}}
\newcommand{\SpecialStringTok}[1]{\textcolor[rgb]{0.13,0.47,0.30}{#1}}
\newcommand{\StringTok}[1]{\textcolor[rgb]{0.13,0.47,0.30}{#1}}
\newcommand{\VariableTok}[1]{\textcolor[rgb]{0.07,0.07,0.07}{#1}}
\newcommand{\VerbatimStringTok}[1]{\textcolor[rgb]{0.13,0.47,0.30}{#1}}
\newcommand{\WarningTok}[1]{\textcolor[rgb]{0.37,0.37,0.37}{\textit{#1}}}

\providecommand{\tightlist}{%
  \setlength{\itemsep}{0pt}\setlength{\parskip}{0pt}}\usepackage{longtable,booktabs,array}
\usepackage{calc} % for calculating minipage widths
% Correct order of tables after \paragraph or \subparagraph
\usepackage{etoolbox}
\makeatletter
\patchcmd\longtable{\par}{\if@noskipsec\mbox{}\fi\par}{}{}
\makeatother
% Allow footnotes in longtable head/foot
\IfFileExists{footnotehyper.sty}{\usepackage{footnotehyper}}{\usepackage{footnote}}
\makesavenoteenv{longtable}
\usepackage{graphicx}
\makeatletter
\newsavebox\pandoc@box
\newcommand*\pandocbounded[1]{% scales image to fit in text height/width
  \sbox\pandoc@box{#1}%
  \Gscale@div\@tempa{\textheight}{\dimexpr\ht\pandoc@box+\dp\pandoc@box\relax}%
  \Gscale@div\@tempb{\linewidth}{\wd\pandoc@box}%
  \ifdim\@tempb\p@<\@tempa\p@\let\@tempa\@tempb\fi% select the smaller of both
  \ifdim\@tempa\p@<\p@\scalebox{\@tempa}{\usebox\pandoc@box}%
  \else\usebox{\pandoc@box}%
  \fi%
}
% Set default figure placement to htbp
\def\fps@figure{htbp}
\makeatother

\makeatletter
\@ifpackageloaded{caption}{}{\usepackage{caption}}
\AtBeginDocument{%
\ifdefined\contentsname
  \renewcommand*\contentsname{Table of contents}
\else
  \newcommand\contentsname{Table of contents}
\fi
\ifdefined\listfigurename
  \renewcommand*\listfigurename{List of Figures}
\else
  \newcommand\listfigurename{List of Figures}
\fi
\ifdefined\listtablename
  \renewcommand*\listtablename{List of Tables}
\else
  \newcommand\listtablename{List of Tables}
\fi
\ifdefined\figurename
  \renewcommand*\figurename{Figure}
\else
  \newcommand\figurename{Figure}
\fi
\ifdefined\tablename
  \renewcommand*\tablename{Table}
\else
  \newcommand\tablename{Table}
\fi
}
\@ifpackageloaded{float}{}{\usepackage{float}}
\floatstyle{ruled}
\@ifundefined{c@chapter}{\newfloat{codelisting}{h}{lop}}{\newfloat{codelisting}{h}{lop}[chapter]}
\floatname{codelisting}{Listing}
\newcommand*\listoflistings{\listof{codelisting}{List of Listings}}
\makeatother
\makeatletter
\makeatother
\makeatletter
\@ifpackageloaded{caption}{}{\usepackage{caption}}
\@ifpackageloaded{subcaption}{}{\usepackage{subcaption}}
\makeatother
\journal{Journal Name}

\usepackage[]{natbib}
\bibliographystyle{elsarticle-harv}
\usepackage{bookmark}

\IfFileExists{xurl.sty}{\usepackage{xurl}}{} % add URL line breaks if available
\urlstyle{same} % disable monospaced font for URLs
\hypersetup{
  pdftitle={Identifying Obstacles to Equitable Flood Mitigation Funding},
  pdfauthor={Julian Plough; Miyuki Hino; Antonia Sebastian; Helena Garcia; Russell Blessing},
  pdfkeywords={Flood mitigation, Climate resilience},
  colorlinks=true,
  linkcolor={blue},
  filecolor={Maroon},
  citecolor={Blue},
  urlcolor={Blue},
  pdfcreator={LaTeX via pandoc}}


\setlength{\parindent}{6pt}
\begin{document}

\begin{frontmatter}
\title{Identifying Obstacles to Equitable Flood Mitigation Funding}
\author[2]{Julian Plough%
%
}

\author[1,2]{Miyuki Hino%
%
}

\author[1,2]{Antonia Sebastian%
%
}

\author[2]{Helena Garcia%
%
}

\author[1]{Russell Blessing%
%
}


\affiliation[1]{organization={University of North Carolina at Chapel
Hill, Department of Earth, Marine and Environmental
Sciences},city={Chapel
Hill},country={USA},countrysep={,},postcodesep={}}
\affiliation[2]{organization={University of North Carolina, Environment,
Ecology and Energy Program},city={Chapel
Hill},country={USA},countrysep={,},postcodesep={}}

\cortext[cor1]{Corresponding author}





        
\begin{abstract}
As damages from floods, fires, and other climate extremes mount, public
investments in risk reduction have also grown. In the US, public
spending on risk reduction efforts totaled over \$X in 2024, and demand
for such funds regularly exceeds available funding. Evidence suggests
that in this competitive landscape for funding, urban and whiter
communities have fared better at obtaining funding but are often
spending it on more disadvantaged neighborhoods within them. However,
the reasons for this trend are unclear: complexities in the application
process, cost-benefit analysis requirements, and flood damage may all
play a role in driving the uneven funding patterns. Here, using a novel
dataset tracking properties from flood exposure to application to
funding receipt, we identify how each hurdle in the mitigation process
shapes the properties that ultimately benefit. Our findings suggest that
application for federal flood mitigation assistance serves as a pinch
point where homeowners in less-white neighborhoods and more affordable
homes relative to their neighbors apply. This trend deepens for those
who ultimately receive funding, while higher flood-exposure has minimal
correlation, implying that funding may not reach all who need it or may
be interested.
\end{abstract}





\begin{keyword}
    Flood mitigation \sep 
    Climate resilience
\end{keyword}
\end{frontmatter}
    

\section{Introduction}\label{introduction}

Impacts from natural hazards in the United States are increasing in cost
and frequency \citep{noaa2024}. While demand and funding for climate
resilience measures has grown significantly in recent years, there are
widespread concerns about access to those funds and the extent to which
they perpetuate or mitigate existing inequities
\citep{fema2023, junod2021, miller2023}. Evidence suggests that while
vulnerable groups including mobile home residents and racial minorities
are over-represented in flood hotspots, wealthier and whiter communities
have received more federal funds to adapt to and recover from floods
\citep{elliot2020, tate2021}. Several explanations for this disparity
have been proposed, including federal cost-benefit analysis
requirements, limited local government capacity to apply for grants, and
lack of household awareness about funding opportunities. However,
research to pinpoint funding obstacles has been constrained by only
observing who ultimately receives funding, not who is eligible or who
applies for funding.As a result, it is unclear what the barriers are,
and therefore which policy or procedural changes would most improve
access to climate resilience funds.

Here, we combine novel datasets on over two decades of flood exposure
and applications for federal funding in North Carolina to examine the
full process: who floods, who applies for mitigation funding, and who
receives it. This flooded-to-funded pipeline (Fig. 1) is critical for
understanding the process by which households experience the Hazard
Mitigation Assistance (HMA) process. By analyzing the household- and
neighborhood-level characteristics at each stage of the process, we
provide new evidence regarding how the application process shapes the
distribution of funding. For example, if households located in
marginalized communities rarely submit applications, the primary issue
may be lack of awareness or resources for applying. In contrast, if they
are applying but not being funded, the selection criteria may be driving
disproportionate access.

Our initial results demonstrate that the application stage is crucial
for understanding the mitigation funding pipeline. 67.5\% of properties
that experience flooding are eligible for public assistance, and 61\%
are in communities where local governments have submitted applications
for assistance. However, only 2.3\% of all flooded properties apply and
\textasciitilde1\% are funded. Properties that apply are significantly
more likely to have experienced more than one flooding event over the
period. Compared to the population of flooded properties, properties
that apply for funding have lower property values (58th percentile to
42nd percentile) and come from census block groups with a lower share of
white residents (74\% to 64\%). Among properties receiving funding,
property values and the share of white residents in the block group are
even lower. These results are consistent with some nationwide findings,
where wealthier and whiter communities have been linked to mitigation
access {[}mach2019; Elliot 2020{]}, however we add fidelity in that
applying and funded properties are also lower-in-value than their
flooded neighbors. Notably, these trends emerge at the newly-visible
application stage and widen for those ultimately funded. We are the
first to demonstrate that unpacking and examining this stage is pivotal
to understanding potential obstacles to equitable mitigation funding.

\section{Bibliography styles}\label{bibliography-styles}

With this template using elsevier class, natbib will be used. Three
bibliographic style files (*.bst) are provided and their use controled by
\texttt{cite-style} option:

\begin{itemize}
\tightlist
\item
  \texttt{citestyle:\ number} (default) will use
  \texttt{elsarticle-num.bst} - can be used for the numbered scheme
\item
  \texttt{citestyle:\ numbername} will use
  \texttt{elsarticle-num-names.bst} - can be used for numbered with new
  options of natbib.sty
\item
  \texttt{citestyle:\ authoryear} will use \texttt{elsarticle-harv.bst}
  --- can be used for author year scheme
\end{itemize}

This \texttt{citestyle} will insert the right \texttt{.bst} and set the
correct \texttt{classoption} for \texttt{elsarticle} document class.

Using \texttt{natbiboptions} variable in YAML header, you can set more
options for \texttt{natbib} itself . Example

\begin{Shaded}
\begin{Highlighting}[]
\FunctionTok{natbiboptions}\KeywordTok{:}\AttributeTok{ longnamesfirst,angle,semicolon}
\end{Highlighting}
\end{Shaded}

\subsection{Using CSL}\label{using-csl}

If \texttt{cite-method} is set to \texttt{citeproc} in
\texttt{elsevier\_article()}, then pandoc is used for citations instead
of \texttt{natbib}. In this case, the \texttt{csl} option is used to
format the references. By default, this template will provide an
appropriate style, but alternative \texttt{csl} files are available from
\url{https://www.zotero.org/styles?q=elsevier}. These can be downloaded
and stored locally, or the url can be used as in the example header.

\section{Equations}\label{equations}

Here is an equation: \[ 
  f_{X}(x) = \left(\frac{\alpha}{\beta}\right)
  \left(\frac{x}{\beta}\right)^{\alpha-1}
  e^{-\left(\frac{x}{\beta}\right)^{\alpha}}; 
  \alpha,\beta,x > 0 .
\]

Inline equations work as well: \(\sum_{i = 2}^\infty\{\alpha_i^\beta\}\)

\section{Figures and tables}\label{figures-and-tables}

Figure~\ref{fig-meaningless} is generated using an R chunk.

\phantomsection\label{cell-fig-meaningless}
\begin{figure}[H]

\centering{

\includegraphics[width=0.5\linewidth,height=\textheight,keepaspectratio]{index_files/figure-pdf/fig-meaningless-1.pdf}

}

\caption{\label{fig-meaningless}A meaningless scatterplot}

\end{figure}%

\textsubscript{Source:
\href{https://russblessing.github.io/Obstacles/index-preview.html}{Article
Notebook}}

\section{Tables coming from R}\label{tables-coming-from-r}

Tables can also be generated using R chunks, as shown in
Table~\ref{tbl-simple} example.

\begin{Shaded}
\begin{Highlighting}[]
\NormalTok{knitr}\SpecialCharTok{::}\FunctionTok{kable}\NormalTok{(}\FunctionTok{head}\NormalTok{(mtcars)[,}\DecValTok{1}\SpecialCharTok{:}\DecValTok{4}\NormalTok{])}
\end{Highlighting}
\end{Shaded}

\begin{longtable}[]{@{}lrrrr@{}}

\caption{\label{tbl-simple}Caption centered above table}

\tabularnewline

\toprule\noalign{}
& mpg & cyl & disp & hp \\
\midrule\noalign{}
\endhead
\bottomrule\noalign{}
\endlastfoot
Mazda RX4 & 21.0 & 6 & 160 & 110 \\
Mazda RX4 Wag & 21.0 & 6 & 160 & 110 \\
Datsun 710 & 22.8 & 4 & 108 & 93 \\
Hornet 4 Drive & 21.4 & 6 & 258 & 110 \\
Hornet Sportabout & 18.7 & 8 & 360 & 175 \\
Valiant & 18.1 & 6 & 225 & 105 \\

\end{longtable}

\textsubscript{Source:
\href{https://russblessing.github.io/Obstacles/index-preview.html}{Article
Notebook}}


\renewcommand\refname{References}
  \bibliography{bibliography.bib}



\end{document}
